\documentclass[12pt,a4paper]{article}

\usepackage[utf8]{inputenc}
\usepackage{geometry}
\usepackage{graphicx}
\usepackage{hyperref}
\usepackage{listings}
\usepackage{xcolor}
\usepackage{amsmath}
\usepackage{amssymb}

\geometry{
    left=3cm,
    right=3cm,
    top=3cm,
    bottom=3cm
}

\hypersetup{
    colorlinks=true,
    linkcolor=black,
    urlcolor=blue
}

\lstset{
    basicstyle=\ttfamily\small,
    keywordstyle=\color{blue}\bfseries,
    commentstyle=\color{gray}\itshape,
    stringstyle=\color{red},
    numbers=left,
    numberstyle=\tiny\color{gray},
    frame=single,
    breaklines=true
}

\pagestyle{headings}

\begin{document}

% COVER PAGE
\begin{titlepage}
    \centering
    
    \vspace*{1cm}
    
    {\LARGE\bfseries Tugas Kecil 1 IF2211 Strategi Algoritma}
    
    \vspace{0.5cm}
    {\Large\bfseries Semester II Tahun 2024/2025}
    
    \vspace{1.5cm}
    
    \includegraphics[width=0.7\textwidth]{../home.png}
    
    \vspace{1.5cm}
    
    {\Huge\bfseries Penyelesaian Permainan Queens}
    
    {\Huge\bfseries dengan Algoritma Brute Force}
    
    \vspace{2cm}
    
    \begin{tabular}{ll}
        \textbf{Nama} & : Made Branenda Jordhy \\[0.3cm]
        \textbf{NIM} & : 13524026 \\[0.3cm]
        \textbf{Program Studi} & : Teknik Informatika \\[0.3cm]
    \end{tabular}
    
    \vfill
    
    {\large Institut Teknologi Bandung}
    
    {\large 2026}
    
\end{titlepage}

% TABLE OF CONTENTS
\tableofcontents
\newpage

% BAB I: PENDAHULUAN
\section{Pendahuluan}

Queens adalah game logic yang ada pada platform LinkedIn, di mana pemain harus menempatkan queen
pada board persegi berwarna dengan aturan tertentu. Each row, column, dan region warna harus
berisi tepat satu queen, dan ga boleh ada dua queen yang bersebelahan (termasuk diagonal).
Game ini merupakan variasi dari masalah klasik N-Queens yang sudah kita kenal, tapi dengan
tambahan constraint berupa region warna yang membuat jadi lebih challenging.

Dalam tugas kecil ini, saya mengimplementasikan solusi menggunakan algoritma pure brute force
tanpa optimisasi apapun. Program dibuat dengan bahasa Go dan dilengkapi dengan antarmuka grafis
(GUI) menggunakan framework Fyne untuk memvisualisasikan proses pencarian solusi secara realtime.
Selain itu, program juga dapat menyimpan hasil solusi dalam bentuk gambar PNG.
Tujuan utama adalah memahami cara kerja algoritma brute force dalam menyelesaikan masalah
pencarian kombinasi, serta menganalisis performanya pada berbagai ukuran board mulai dari
4x4 hingga 11x11.

\newpage

% BAB II: DASAR TEORI
\section{Dasar Teori}

\subsection{Algoritma Brute Force}
Algoritma brute force adalah cara paling sederhana untuk menyelesaikan masalah 
"coba semua kemungkinan yang ada sampai ketemu solusinya". Ide dasarnya sangat straightforward
kita generate semua kombinasi yang mungkin, lalu cek satu per satu apakah kombinasi tersebut
valid atau tidak. Jika valid, yaa itu jawabannya. Kalau tidak, lanjut ke kombinasi berikutnya.

Keuntungan utama dari approach brute force adalah kesederhanaannya. Kita tidak perlu mikir
strategi rumit atau trik khusus, kalau kata Mr. Rila sih ini algoritma ga mikir,
tinggal exhaustive search aja sampai ketemu. Plus, kalau memang
ada solusinya, dijamin pasti ketemu karena kita cek semua kemungkinan. Tapi ya kelemahannya
juga jelas, akan lambat jika jumlah kombinasinya banyak, dalam case game ini $N > 9$.
Kompleksitas waktunya biasanya eksponensial atau faktorial, jadi tidak cocok untuk problem skala besar.

\subsection{Kompleksitas untuk Permainan Queens}
Dalam permainan Queens, jumlah kombinasi yang harus dicoba sangat bergantung pada ukuran board
dan distribusi region-nya. Misalnya untuk board N×N dengan N region, kalau setiap region punya N
kotak, maka total kombinasi yang mungkin adalah $N^N$. Sebagai gambaran:

\begin{itemize}
    \item Board 4×4: sekitar 256 kombinasi (masih cepat, di bawah 1 milidetik)
    \item Board 7×7: sekitar 800 ribu kombinasi (beberapa milidetik)
    \item Board 9×9: bisa mencapai 387 juta kombinasi (beberapa detik)
    \item Board 11×11 ke atas: miliaran kombinasi (bisa lama banget)
\end{itemize}


\newpage


% BAB III: ALGORITMA DAN IMPLEMENTASI
\section{Algoritma dan Implementasi}

\subsection{Cara Kerja Algoritma}

Implementasi yang saya buat di sini menggunakan pendekatan Cartesian Product untuk generate semua
kombinasi posisi queen yang mungkin, setelah itu baru cek satu per satu validitasnya.

\subsubsection{Langkah-Langkah}

\textbf{1. Read dan Validate Input}

Pertama-tama, program baca file input yang isinya konfigurasi board.
Setiap huruf (A, B, C, dst) merepresentasikan satu region warna. Program akan cek dulu
apakah board-nya berbentuk persegi dan jumlah region unique-nya sesuai dengan ukuran board.
Kalau ada yang salah, langsung rejected.

\textbf{2. Generate Semua Kombinasi}

Ini bagian yang paling crucial karena konsep brute force ada di sini.
Untuk setiap region, ambil semua kotak yang ada di dalamnya.
Misalnya region A punya 5 kotak, region B punya 6 kotak, dan seterusnya.
Setelah itu kita generate Cartesian Product dari semua region ini.
Jadi jika ada 9 region, kita kombinasikan semua kemungkinan. Pilih 1 kotak dari region A,
1 dari region B, sampai 1 dari region I. Total kombinasinya adalah perkalian jumlah kotak
di setiap region.

\textbf{3. Coba Satu per Satu}

Setelah punya semua kombinasi, kita coba satu per satu. Untuk setiap kombinasi:
\begin{itemize}
    \item Cek apakah ada 2 queen di baris yang sama (tidak boleh)
    \item Cek apakah ada 2 queen di kolom yang sama (tidak boleh)
    \item Cek apakah ada 2 queen yang bersebelahan, termasuk diagonal (tidak boleh)
\end{itemize}

Kalau ketemu kombinasi yang valid, langsung stop dan return hasilnya. Kalau tidak, lanjut ke kombinasi berikutnya. Setiap 50 iterasi, program ngirim update ke GUI supaya kita bisa lihat progressnya.

\textbf{4. Hasil Akhir}

Kalau ketemu solusi, program return posisi semua queen beserta statistiknya (berapa iterasi dan berapa lama). Kalau sampai habis semua kombinasi belum ketemu, berarti memang tidak ada solusi untuk konfigurasi board tersebut.

\subsubsection{Gambaran Kode}

Secara garis besar, implementasinya seperti ini:

\begin{lstlisting}[language=Python, caption=Gambaran Algoritma (Pseudocode)]
function SolveQueens(board):
    regions = ExtractRegions(board) 
    allCombinations = GenerateAllCombinations(regions)  
    iterations = 0
    startTime = now()
    
    for each combination in allCombinations:
        iterations++
        
        if iterations mod 50 == 0:
            SendUpdateToGUI(combination, iterations)
        
        if IsValid(combination):
            return Solution(combination, iterations, elapsed_time)
    
    return NoSolution()

function IsValid(queens):
    for i = 0 to n-1:
        for j = i+1 to n-1:
            if queens[i].row == queens[j].row:
                return false
            if queens[i].col == queens[j].col:
                return false
            if abs(queens[i].row - queens[j].row) <= 1 and 
               abs(queens[i].col - queens[j].col) <= 1:
                return false
    
    return true 
\end{lstlisting}

\subsection{Struktur Data}

Untuk implementasinya di Go, saya pakai beberapa struct:

\textbf{1. Cell} - save posisi satu kotak di board
\begin{lstlisting}[caption=Struct Cell]
type Cell struct {
    Row int
    Col int 
}
\end{lstlisting}

\textbf{2. Region} - save satu region warna lengkap
\begin{lstlisting}[caption=Struct Region]
type Region struct {
    Letter rune      // huruf region (A, B, C, dst, bisa tdk consecutive)
    Cells  []Cell    // semua kotak yang masuk region ini
}
\end{lstlisting}

\textbf{3. Board} - save seluruh board
\begin{lstlisting}[caption=Struct Board]
type Board struct {
    Size    int         //(NxN)
    Grid    [][]rune    //matrix berisi huruf region
    Regions []Region    // list semua region yg ada
}
\end{lstlisting}

\textbf{4. SolverResult} - save result
\begin{lstlisting}[caption=Struct Solver Result]
type SolverResult struct {
    Solution      []Cell           //posisi semua queens
    Iterations    int       
    ExecutionTime time.Duration
    Found         bool
}
\end{lstlisting}

\subsection{Pure Brute Force tanpa Optimisasi}

Sesuai spesifikasi tugas, algoritma ini sengaja dibuat pure brute force tanpa optimisasi apapun. Artinya saya \textbf{TIDAK} pakai:
\begin{itemize}
    \item \textbf{Backtracking}: Biasanya langsung stop/block begitu ketemu constraint yang dilanggar
    \item \textbf{Forward Checking}: Mengurangi pilihan sebelum dicoba
    \item \textbf{Constraint Propagation}: Constraint untuk pruning
    \item \textbf{Heuristic}
\end{itemize}

\newpage

% BAB IV: HASIL PENGUJIAN
\section{Hasil Pengujian}

Program telah diuji dengan 10 test cases berbeda dengan ukuran dan kompleksitas yang bervariasi.
Setiap test case ada dua screensho yaitu input dan outpit.

\subsection{Test Case 1: 9×9 Board}
\textbf{File}: \texttt{test1\_9x9.txt} \\
\textbf{Ukuran}: 9×9 \\
\textbf{Region}: 9 regions (A-I)

\textbf{Input:}
\begin{center}
\begin{lstlisting}
AAABBCCCD
ABBBBCECD
ABBBDCECD
AAABDCCCD
BBBBDDDDD
FGGGDDHDD
FGIGDDHDD
FGIGDDHDD
FGGGDDHHH
\end{lstlisting}
\end{center}

\textbf{Output:}
\begin{center}
    \includegraphics[width=0.6\textwidth]{media/test1.png}
\end{center}

\subsection{Test Case 2: 8×8 Board}
\textbf{File}: \texttt{test2\_8x8.txt} \\
\textbf{Ukuran}: 8×8 \\
\textbf{Region}: 8 regions (A-H)

\textbf{Input:}
\begin{center}
\begin{lstlisting}
AAABBBBB
AAABCCDD
AAAAAADD
EEAAAAAD
FEAAAAAD
FEEAAAAA
FGGGHAAA
FFFGHAAA
\end{lstlisting}
\end{center}

\textbf{Output:}
\begin{center}
    \includegraphics[width=0.6\textwidth]{media/test2.png}
\end{center}

\subsection{Test Case 3: 7×7 Board}
\textbf{File}: \texttt{test3\_7x7.txt} \\
\textbf{Ukuran}: 7×7 \\
\textbf{Region}: 7 regions (A-G)

\textbf{Input:}
\begin{center}
\begin{lstlisting}
ABBCCDD
AABBCDD
ABBBBEE
BBBBBBE
BBFBBBB
BFFFBBB
GGGGGBB
\end{lstlisting}
\end{center}

\textbf{Output:}
\begin{center}
    \includegraphics[width=0.6\textwidth]{media/test3.png}
\end{center}

\subsection{Test Case 4: 6×6 Board}
\textbf{File}: \texttt{test4\_6x6.txt} \\
\textbf{Ukuran}: 6×6 \\
\textbf{Region}: 6 regions (A-F)

\textbf{Input:}
\begin{center}
\begin{lstlisting}
AAAAAA
AAAAAA
AAAAAA
AAAAAA
AAAAAA
ABCDEF
\end{lstlisting}
\end{center}

\textbf{Output:}
\begin{center}
    \includegraphics[width=0.6\textwidth]{media/test4.png}
\end{center}

\subsection{Test Case 5: 5×5 Board}
\textbf{File}: \texttt{test5\_5x5.txt} \\
\textbf{Ukuran}: 5×5 \\
\textbf{Region}: 5 regions (A-E)

\textbf{Input:}
\begin{center}
\begin{lstlisting}
AABBB
AABCB
AABCC
DDBCC
DDEEE
\end{lstlisting}
\end{center}

\textbf{Output:}
\begin{center}
    \includegraphics[width=0.6\textwidth]{media/test5.png}
\end{center}

\subsection{Test Case 6: 4×4 Board}
\textbf{File}: \texttt{test6\_4x4.txt} \\
\textbf{Ukuran}: 4×4 \\
\textbf{Region}: 4 regions (A-D)

\textbf{Input:}
\begin{center}
\begin{lstlisting}
AABB
ABBB
CCDB
CCDD
\end{lstlisting}
\end{center}

\textbf{Output:}
\begin{center}
    \includegraphics[width=0.6\textwidth]{media/test6.png}
\end{center}

\subsection{Test Case 7: 10×10 Board}
\textbf{File}: \texttt{test7\_10x10.txt} \\
\textbf{Ukuran}: 10×10 \\
\textbf{Region}: 10 regions (A-K)

\textbf{Input:}
\begin{center}
\begin{lstlisting}
AABBCCDDEE
AABBCCDDEE
FFBBCCDDGG
FFHHCCIIGG
FFHHJJIIGG
FFHHJJIIGG
FFHHJJIIGG
KKHHJJIIGG
KKHHJJIIGG
KKHHJJIIGG
\end{lstlisting}
\end{center}

\textbf{Output:}
\begin{center}
    \includegraphics[width=0.6\textwidth]{media/test7.png}
\end{center}

\subsection{Test Case 8: 11×11 Board}
\textbf{File}: \texttt{test8\_11x11.txt} \\
\textbf{Ukuran}: 11×11 \\
\textbf{Region}: 11 regions (A-K)

\textbf{Input:}
\begin{center}
\begin{lstlisting}
AAABBCCCCCC
ADDBBBCCCCC
DDDBEECCCCC
DDDBEEFCCCC
DDDBEEFFGGG
DDDHHFFFGGG
DDDHIIFFGGG
JJJHIIFFGGG
JJJHIIKKGGG
JJJHIIKKGGG
JJJHIIKKGGG
\end{lstlisting}
\end{center}

\textbf{Output:}
\begin{center}
    \includegraphics[width=0.6\textwidth]{media/test8.png}
\end{center}

\subsection{Test Case 9: 26×26 Board (Extreme)}
\textbf{File}: \texttt{test9\_26x26.txt} \\
\textbf{Ukuran}: 26×26 \\
\textbf{Region}: 26 regions (A-Z) \\
\textit{Catatan: Test case ekstrem untuk menguji batas algoritma}

\textbf{Input:}
\begin{center}
\begin{lstlisting}[basicstyle=\ttfamily\tiny]
ABCDEFGHIJKLMNOPQRSTUVWXYZ
ABCDEFGHIJKLMNOPQRSTUVWXYZ
ABCDEFGHIJKLMNOPQRSTUVWXYZ
ABCDEFGHIJKLMNOPQRSTUVWXYZ
ABCDEFGHIJKLMNOPQRSTUVWXYZ
ABCDEFGHIJKLMNOPQRSTUVWXYZ
ABCDEFGHIJKLMNOPQRSTUVWXYZ
ABCDEFGHIJKLMNOPQRSTUVWXYZ
ABCDEFGHIJKLMNOPQRSTUVWXYZ
ABCDEFGHIJKLMNOPQRSTUVWXYZ
ABCDEFGHIJKLMNOPQRSTUVWXYZ
ABCDEFGHIJKLMNOPQRSTUVWXYZ
ABCDEFGHIJKLMNOPQRSTUVWXYZ
ABCDEFGHIJKLMNOPQRSTUVWXYZ
ABCDEFGHIJKLMNOPQRSTUVWXYZ
ABCDEFGHIJKLMNOPQRSTUVWXYZ
ABCDEFGHIJKLMNOPQRSTUVWXYZ
ABCDEFGHIJKLMNOPQRSTUVWXYZ
ABCDEFGHIJKLMNOPQRSTUVWXYZ
ABCDEFGHIJKLMNOPQRSTUVWXYZ
ABCDEFGHIJKLMNOPQRSTUVWXYZ
ABCDEFGHIJKLMNOPQRSTUVWXYZ
ABCDEFGHIJKLMNOPQRSTUVWXYZ
ABCDEFGHIJKLMNOPQRSTUVWXYZ
ABCDEFGHIJKLMNOPQRSTUVWXYZ
ABCDEFGHIJKLMNOPQRSTUVWXYZ
\end{lstlisting}
\end{center}

\textbf{Output:}
\begin{center}
    \includegraphics[width=0.6\textwidth]{media/test9.png}
\end{center}

\subsection{Test Case 10: 6×7 Board (Non-Square)}
\textbf{File}: \texttt{test10\_6x7.txt} \\
\textbf{Ukuran}: 6×7 (non-square) \\
\textbf{Region}: 6 regions (A-F) \\
\textit{Catatan: Test case untuk board non-persegi (akan ditolak oleh validator)}

\textbf{Input:}
\begin{center}
\begin{lstlisting}
ABCDEF
ABCDEF
ABCDEF
ABCDEF
ABCDEF
ABCDEF
ABCDEF
\end{lstlisting}
\end{center}

\textbf{Output:}
\begin{center}
    \includegraphics[width=0.6\textwidth]{media/test10.png}
\end{center}

\subsection{Analisis Performa}

\begin{center}
\begin{tabular}{|c|c|c|c|c|}
\hline
\textbf{Test} & \textbf{Ukuran} & \textbf{Status} & \textbf{Iterasi} & \textbf{Waktu (ms)} \\ \hline
Test 1 & 9×9 & success & 26,880,000 & 16190 ms \\ \hline
Test 2 & 8×8 & success & 1,036,800 & 165 ms \\ \hline
Test 3 & 7×7 & success & 74,880 & 18 ms \\ \hline
Test 4 & 6×6 & failed & - & - \\ \hline
Test 5 & 5×5 & success & 2,520 & 0 ms (too small) \\ \hline
Test 6 & 4×4 & success & 216 & 0 ms (too small) \\ \hline
Test 7 & 10×10 & failed & - & - \\ \hline
Test 8 & 11×11 & unknown & unknown & unknown (terlalu lama jadi author interrupt) \\ \hline
Test 9 & 26×26 & unknown & unknown & unknown (board melebihi batas screen, unhandled) \\ \hline
Test 10 & 6×7 & failed & - & - \\ \hline
\end{tabular}
\end{center}

\textbf{Observasi}:
\begin{itemize}
    \item Waktu eksekusi meningkat drastis seiring bertambahnya ukuran board
    \item Board kecil (4×4 - 6×6) sangat cepat, dalam hitungan milidetik
    \item Board menengah (7×7 - 9×9) butuh beberapa detik
    \item Board besar (10×10 ke atas) bisa sangat lama atau timeout
    \item Distribusi region sangat berpengaruh pada jumlah kombinasi
    \item Live update visualization membantu tracking progress
\end{itemize}

\newpage

% BAB VI: KESIMPULAN
\section{Kesimpulan}

Setelah mengimplementasikan dan testing game Queens Solver dengan algoritma brute force murni,
ada beberapa hal yang bisa saya simpulkan. Pertama, algoritma brute force memang berhasil
menyelesaikan permainan Queens. Walaupun caranya sangat straightforward
(generate semua kombinasi lalu cek satu-satu), tapi ya... it works.
Jika ada solusinya, pasti ketemu. Cuma sangat lambat jika size board (N) besar.
Kedua, kompleksitas eksponensialnya bener-bener teras. Board 4×4 bisa selesai dalam
beberapa milisecond, tapi begitu naik ke 9×9 udah butuh beberapa detik.
Apalagi jika 10×10 ke atas, bisa lama banget (dalam kasus saya seperti test-8) atau bahkan timeout.
Ini yang bikin brute force kurang praktis untuk problem skala besar.
Ketiga, validasi input itu lumayan crucial. Jika tidak dicek dari awal, bisa-bisa program crash di
tengah jalan karena input yang tidak valid. Better fail fast daripada buang-buang waktu compute.

\newpage

% DAFTAR PUSTAKA
\section{Daftar Pustaka}

\begin{enumerate}
    \item Munir, Rinaldi. (2026). \textit{Algoritma Brute Force (Bagian 1)}. Program Studi Teknik Informatika, Sekolah Teknik Elektro dan Informatika, Institut Teknologi Bandung. \\
    \url{https://informatika.stei.itb.ac.id/~rinaldi.munir/Stmik/2025-2026/02-Algoritma-Brute-Force-(2026)-Bag1.pdf}
    
    \item Munir, Rinaldi. (2026). \textit{Algoritma Brute Force (Bagian 2)}. Program Studi Teknik Informatika, Sekolah Teknik Elektro dan Informatika, Institut Teknologi Bandung. \\
    \url{https://informatika.stei.itb.ac.id/~rinaldi.munir/Stmik/2025-2026/03-Algoritma-Brute-Force-(2026)-Bag2.pdf}
    
    \item LinkedIn. (2024). \textit{Queens - LinkedIn Games}. \\
    \url{https://www.linkedin.com/games/queens/}
    
    \item The Go Programming Language. (2024). \textit{Go Documentation}. \\
    \url{https://golang.org/doc/}
    
    \item Fyne.io. (2024). \textit{Fyne Toolkit Documentation - Cross Platform GUI Framework}. \\
    \url{https://fyne.io/}
\end{enumerate}

\newpage

% LAMPIRAN
\section{Lampiran}

\subsection{Checklist Pemenuhan Kriteria}

\begin{center}
\begin{tabular}{|c|p{8cm}|c|c|}
\hline
\textbf{No} & \textbf{Poin} & \textbf{Ya} & \textbf{Tidak} \\ \hline
1 & Program berhasil dikompilasi tanpa kesalahan & \checkmark & \\ \hline
2 & Program berhasil dijalankan & \checkmark & \\ \hline
3 & Solusi yang diberikan program benar dan mematuhi aturan permainan & \checkmark & \\ \hline
4 & Program dapat membaca masukan berkas .txt serta menyimpan solusi dalam berkas .txt & \checkmark & \\ \hline
5 & Program memiliki Graphical User Interface (GUI) & \checkmark & \\ \hline
6 & Program dapat menyimpan solusi dalam bentuk file gambar & \checkmark & \\ \hline
\end{tabular}
\end{center}

\vspace{1cm}

\subsection{Repository Source Code}

Source code lengkap beserta dokumentasi dan test cases dapat diakses di repository GitHub:


\begin{center}
\Large
\url{https://github.com/ethj0r/Tucil1_13524026}
\end{center}

\vspace{1cm}



\vspace{2cm}

\textbf{Pernyataan:} Tugas ini disusun sepenuhnya tanpa bantuan kecerdasan buatan (Generative AI),
melainkan hasil pemikiran dan analisis mandiri.

\vspace{1cm}

\begin{center}
\includegraphics[width=0.1\textwidth]{media/signature.png}
\end{center}

\begin{center}
\textbf{Made Branenda Jordhy} \\
\textbf{13524026}
\end{center}

\end{document}
